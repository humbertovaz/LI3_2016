\chapter{Conclusão}

A implementação de um programa de gestão de hipermercados em Java provou ser mais fácil
quando comparado com a implementação em C. Tal deve-se, em grande maioria, à necessidade de definir de raiz um grande número de funções como por exemplo estruturas. Esta vantagem de implementaçãoo permite desenvolver soluções adicionais que teriam um custo considerável em C.
Assim sendo foi possível concretizar as várias queries com maior facilidade, sem prejudicar a sua velocidade, graças à facilidade em trocar e testar diferentes estruturas, desde que estas partilhem da mesma API.

A modularidade foi garantida através da criação de classes com API completa, com construtores apropriados e métodos que permitem um bom número de operações com as classes e módulos do programa. Nesse espírito, foram ainda incluidos não só nas classes dos módulos como em todas as que tal se justificava, métodos essenciais a qualquer classe “bem comportada” nomeadamente toString(), equals() e clone().
O encapsulamento dos módulos foi garantido através do uso de clone’s ao inserir nas estruturas e ao serem procurados elementos nas mesmas.
Além da modularidade e encapsulamento, uma boa estruturação do programa e legibilidade do mesmo também foram aspectos tidos em conta, o que levou à criação de tipos enumerados e métodos e variáveis com nomes sugestivos, ainda que isso tenha levado a que esses nomes fossem por vezes longos.
Os tempos de resposta às queries e de leitura dos ficheiros são também bastante satisfatórios tendo em conta a quantidade de dados a analisar.
Tivemos problemas a implementar a querie 5, e nas queries  estatísticas 1.2 não temos os metodos para invocar na classe hipermercado, para posteriormente a classe gereVendas invocar esse método. 

