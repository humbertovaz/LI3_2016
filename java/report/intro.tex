\chapter{Introdução}

No âmbito da unidade curricular de Laboratórios de Informática III do 2ºano da licenciatura de Engenharia Informática, foi proposto o desenvolvimento de um projeto em linguagem Java, que tem por objectivo ajudar à consolidação dos conteúdos teóricos e práticos e enriquecer os conhecimentos adquiridos na UC de Programação Orientada aos Objectos.

O Projeto, denominado GereVendas, baseia-se num programa de gestão de hipermercados o qual depende de uma lista de clientes, uma lista de produtos e uma lista de compras efetuadas. Este projeto considera-se um grande desafio para nós pelo facto de passarmos a realizar programação em grande escala, uma vez que se trata de grandes volumes de dados e por isso uma maior complexidade.

Como primeira etapa, o mais importante é definir as estruturas utilizadas no trabalho. Assim sendo iremos explicar ao longo do relatório detalhadamente cada uma e os aspetos mais importantes que definem a base do nosso projeto.


Serão explicadas as decisões quanto à resolução das diversas consultas (queries), os resultados obtidos para cada uma destas. Englobaremos também algumas estatísticas e alguns testes de “performance” do nosso programa em relação aos tipos de estruturas de dados oferecidas pela linguagem. Estes testes serão baseados na alteração das nossas estruturas iniciais e serão comparados através dos tempos de execução da leitura e de algumas queries. 




