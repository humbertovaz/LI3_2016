\chapter{Introdução}

No âmbito da unidade curricular de Laboratórios de Informática III do 2ºano do curso de MIEI,  foi proposto o desenvolvimento de um projeto em linguagem C que tem por objetivo ajudar à consolidação dos conteúdos teóricos e práticos e enriquecer os conhecimentos adquiridos nas UCs de Programação Imperativa, de Algoritmos e Complexidade, e da disciplina de Arquitetura de Computadores. 

O Projeto, denominado GereVendas, baseia-se num programa de gestão de hipermercados com 3 filiais que dependem de uma lista de clientes, uma lista de produtos e uma lista de vendas efetuadas.  Cada uma destas listas estará num ficheiro .txt e para cada um dos ficheiros o programa percorre o ficheiro, executando operações que permitam guardar estes dados em memória. Para ajudar nesta tarefa repartir-se-á as tarefas em quatro módulos
módulos. Estes módulos são: um catálogo de clientes; um catálogo de produtos; um módulo de faturação global; um módulo de gestão de filial.

 De forma a preservar o encapsulamento de dados será disponibilizada uma API de
forma a que o utilizador apenas possa aceder através destas funções públicas. Depois dos ficheiros serem carregados o utilizador será capz de executar uma lista de queries previamente fornecida pela equipa docente. Para responder às diferentes queries utilizam-se as funções definidas nas API dos diferentes módulos. 

Este projeto considera-se um grande desafio, pelo facto de passarmos a realizar programação em grande escala, uma vez que se tratam de grandes volumes de dados e por isso uma maior complexidade. Nesse sentido, o desenvolvimento deste programa será realizado à luz dos princípios da modularidade (divisão do código fonte em unidades separadas coerentes) e do encapsulamento (garantia de proteção e acessos controlados aos dados). 