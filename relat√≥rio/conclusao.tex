\chapter{Conclusão}

Uma vez que se tratou de um trabalho de uma dimensão já considerável comparando com o que estávamos habituados envolveu utilização de técnicas particulares e tivemos sempre como objetivo que este trabalho fosse concebido de modo a que seja facilmente modificável, e seja, apesar da complexidade, o mais optimizado possível a todos os níveis.

Inicialmente, tivemos dificuldades nas AVLs pois estávamos a fazer uma AVL para cada módulo. Depois de alguns problemas com o seu balanceamento, acabamos por apostar na utilização da biblioteca standard AVL da GNU, que nos facilitou não só o carregamento dos ficheiros em memória, mas também na realização de algumas queries, devido ao vasto conjunto de úteis funções que a biblioteca contém, evitando assim a repetição de código. 

